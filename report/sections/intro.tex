The aim of this project was to produce a program to solve linear systems using the Jacobi method.

Three different variants of the algorithms are provided.
The first variant is the straightforward implementation of the sequential algorithm.
The second and the third are parallel implementations.
One uses C++11 threads to parallelize the algorithm and distribute the work among different threads, the other employs FastFlow and the \verb|parallel_for| construct.

All the experiments were conducted on a machine using a \emph{Intel Xeon E2650} CPU ($8$ cores clocked at $2$~\si{\giga\hertz} each with $2$ contexts) and a \emph{Intel Xeon Phi} co-processor ($60$ cores clocked at $1$~\si{\giga\hertz} each with $4$ contexts).
For each experiment (i.e.\ for each different implementation and for different linear system sizes) we collected the latency and for parallel versions also computed speedup, the scalability, the efficiency, and the ``best configuration'' that minimizes the latency.
We also produced some graphs to compare expected results against empirical ones.

\paragraph{Summary.} The next section discusses the details of the design, including an analysis of the expected performance of the sequential and parallel implementations.
Section~\ref{sec:implementation} reports some details about the implementation, discussing the classes design, their methods, and some optimization.
Section~\ref{sec:experiments} first introduces the methodology used in the experiments then reports and discuss the experimental results.
Section~\ref{sec:conclusions} concludes the report with some final remarks.
Finally, a brief user manual is reported in Section~\ref{sec:guide}.