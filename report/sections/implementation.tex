The actual implementation consists in three different variants of the Jacobi iterative method, following the ideas in Section~\ref{sec:design}.

The source code is organized as follows:
\begin{itemize}
	\item The main function, in file \verb|main.cpp|, implements the command line parameter parsing, I/O and initialization.
	\item The class \verb|JacobiReport| collects statistics about the execution of the algorithm.
	\item The class \verb|JacobiSolver| implements, together with class \verb|JacobiSequentialSolver|, \verb|JacobiFFSolver|, and \verb|JacobiThreadSolver| a template method pattern.
	Each of the concrete sub-classes implements a \verb|deltax| method specifying how the new approximation of the solution $x$ to the system is computed.
	\item Class \verb|JacobiSequentialSolver| implements \verb|deltax| exactly as in Algorithm~\ref{alg:sequantialjacobi}.
	\item Class \verb|JacobiFFSolver| implements \verb|deltax| using the \verb|parallel_for| building block from FastFlow library; \verb|parallel_for| automatically parallelizes the given loop using a grain size provided as argument to the object's constructor.
	\item Class \verb|JacobiThreadSolver| implements \verb|deltax| using C++11 threads. The implementation is na\"ive, meaning that the threads are re-created at each method invocation (i.e.\ $T_{setup}$ may become non negligible)  and the collector is a simple barrier implemented using \verb|join| method from \verb|thread| class.
\end{itemize}

Please note that, the code of each \verb|deltax| (and also other methods) have been made vectorizable using the feedback provided by the compiler.
Specifically the \verb|if| construct present in the inner-loop in the algorithms above have been removed and the loop have been split in two parts, i.e.\ the first part in range $[0, i)$ and the second $[i+1, N)$ (or  $[i+1, M)$ in case of parallel implementations).